\documentclass{book}
\usepackage[utf8]{inputenc}
\usepackage[english, russian]{babel}
\usepackage[a6paper, landscape, margin=10mm]{geometry}
%\usepackage{fontspec}
%\defaultfontfeatures[\rmfamily, \sffamily]{Ligatures={TeX}}
%\setmainfont[
%	BoldFont={[PTF75F.ttf]}, 
%	BoldItalicFont={[PTF76F.ttf]}, 
%	ItalicFont={[PTF56F.ttf]}, 
%	Ligatures={TeX,Historic}]
%	{[PTF55F.ttf]}
%\setsansfont[
%	BoldFont={[PTS75F.ttf]}, 
%	BoldItalicFont={[PTS76F.ttf]}, 
%	ItalicFont={[PTS56F.ttf]}, 
%	Ligatures={TeX,Historic}]
%	{[PTS55F.ttf]} %% задаёт шрифт без засечек
%\setmonofont[Scale=1.05]{consola.ttf}              %% задаёт моноширинный шрифт 
%\usepackage{unicode-math}
%\setmathfont{[latinmodern-math.otf]}
%\setmathfont[range=\mathit/{latin,Latin}]{[PTF56F.ttf]}
%\setmathfont[range=\mathit/{greek,Greek}]{latinmodern-math.otf}    
%\setmathfont[range=up]{[PTF55F.ttf]}
%\setmathfont[range="2264-"2265]{[PTF55F.ttf]}
%\setmathfont[range="003C-"003E]{[PTF55F.ttf]}
%\allsectionsfont{\sffamily}
\usepackage{indentfirst}
\pagestyle{empty}

\begin{document}

\section*{P-01 Саморегистрация}
\textbf{Участник} должен иметь возможность зарегистрироваться в системе, указав свои 
данные, если это разрешено администратором.

\newpage \section*{P-02 Сумма баллов}
\textbf{Участник} может видеть свой максимальный балл по каждой задаче контеста и,
если разрешено, таблицу баллов для всех участников.

\newpage \section*{P-03 Баллы каждой посылки}
\textbf{Участник} может в зависимости от настроек подзадачи видеть:
\begin{itemize}\setlength{\itemsep}{0pt}
\item общий балл по подзадаче.
\item общий балл по подзадаче либо номер и результат выполнения первого непройденного теста.
\item результат выполнения по каждому тесту.
\end{itemize}

\newpage \section*{P-04 Отправка решения}
\textbf{Участник} может отправить исходный код своего решения в виде файла или ввести 
его на сайте. При этом он должен указать язык программирования (по-умолчанию
на странице выбран язык программирования последней отправки).

\newpage \section*{P-05 Страница контеста }
\textbf{Участник} может получать на странице контеста список задач, информацию о 
максимальном балле по каждой задаче, а также (?) файл с условиями и файл (?) с 
правилами проведения.

\newpage \section*{P-06 Список контестов }
\textbf{Участник} может выбрать контест из списка доступных ему, при этом ему видно 
время, оставшееся до окончания контеста, либо дата/время окончания для 
завершившихся.

\newpage \section*{P-07 Сброс пароля}
\textbf{Участник} может сбросить свой пароль, получив ссылку сброса на email.

\newpage \section*{P-08 Вопросы в контесте}
\textbf{Участник} может задавать вопросы организаторам контеста, в котором он участвует.

\newpage \section*{P-09 Подглядывать в тесты}
\textbf{Участник} может в зависимости от настроек подзадачи в контесте посмотреть
содержание теста, ответа своей посылки, коды выхода своей посылки и чекера, 
затраченное время и память.

% Автор задачи

\newpage \section*{A-01 Создание и редактирование задач}
\textbf{Автор} может создавать, просматривать, редактировать и удалять задачи. 
\textbf{Автор} должен указать её название.
\textbf{Автор} может указать теги для быстрого поиска.
Система должна автоматически фиксировать дату её модификации.

\newpage \section*{A-02 Условия задачи}
При редактировании задачи \textbf{автор} может ввести:
\begin{itemize}\setlength\itemsep{0pt}
	\item текст условия;
	\item формат входных данных;
	\item формат выходных данных;
	\item примечания;
	\item описание системы оценивания;
	\item разбор;
	\item теги;
\end{itemize}

При этом можно ссылаться на файлы задачи.

\newpage \section*{A-03 Файлы задачи}
\textbf{Автор} задачи может добавлять, просматривать, скачивать и удалять файлы задачи.
При этом он может указать тип файла:
\begin{itemize}\setlength{\itemsep}{0pt}
	\item иллюстрация
	\item решение (одно на задачу)
	\item чекер (один на задачу)
	\item генератор тестов
	\item валидатор тестов
\end{itemize}

\newpage \section*{A-04 Подзадачи}
\textbf{Автор} может создавать, просматривать и удалять подзадачи. 

\newpage \section*{A-05 Способ начисления баллов}
\textbf{Автор} должен выбрать для каждой подзадачи способ начисления баллов:
\begin{itemize}\setlength{\itemsep}{0pt}
	\item все или ничего
	\item пропорционально (независимо за каждый тест)
\end{itemize}

\newpage \section*{A-06 Способ отображения результатов}
\textbf{Автор} должен выбрать для каждой подзадачи способ отображения результатов:
\begin{itemize}\setlength{\itemsep}{0pt}
	\item общий балл за подзадачу
	\item общий балл за подзадачу либо номер первого непройденного теста с результатом его проверки
	\item общий балл за подзадачу и результат проверки на каждом тесте
\end{itemize}

\newpage \section*{A-07 Лимиты времени и памяти}
\textbf{Автор} должен указать:
\begin{itemize}\setlength{\itemsep}{0pt}
	\item лимит времени
	\item лимит на используемую память
\end{itemize}

\newpage \section*{A-08 Тесты}
\textbf{Автор} может загружать, скачивать и удалять входные данные тестов.
\textbf{Автор} может создать, просмотреть и отредактировать первые 512 байт входных данных теста.

\newpage \section*{A-09 Генерация входных данных тестов}
\textbf{Автор} может сгенерировать входные данные тестов с помощью генератора тестов.
Для этого он должен написать в системе скрипт генерации, в котором могут быть строки вида:
\begin{itemize}\setlength{\itemsep}{0pt}
	\item \texttt{генератор [параметры] > 1}
	\item \texttt{генератор [параметры] > \$}
	\item \texttt{генератор [параметры] > \{индексы\}}
\end{itemize}
где \texttt{генератор} — это имя генератора тестов, \texttt{параметры} — необязательные параметры, \texttt{\$} — автоматически подставляемый номер теста, а \texttt{индексы} — диапазон индексов номеров тестов.

\newpage \section*{A-10 Генерация выходных данных тестов}
\textbf{Система} должна сгенерировать выходные данные тестов (если таковых не существует) при валидации задачи.

\newpage \section*{A-11 Копирование задач}
\textbf{Автор} может создать копию любой задачи к которой он имеет доступ.

\newpage \section*{A-12 Просмотр списка и поиск задач}
\textbf{Автор} может просматривать список задач:
\begin{itemize}\setlength{\itemsep}{0pt}
	\item свои задачи
	\item задачи контестов в которых он участвует 
\end{itemize}

\textbf{Автор} может сортировать и фильтровать задачи:
\begin{itemize}\setlength{\itemsep}{0pt}
	\item по названию
	\item по автору
	\item по тегам
	\item по дате модификации 
\end{itemize}

\newpage \section*{A-13 Сверкающее будущее}
\textbf{Автор} импортировать / экспортировать задачи. Должны импортироваться задачи в 
формате Poligon.


% Организатор контеста

\newpage \section*{O-xx Отладочная информация}
\textbf{Организатор контеста} может разрешить просмотр отладочной информации по задаче:
\begin{itemize}\setlength{\itemsep}{0pt}
	\item входные данные теста
	\item выходные данные теста
	\item выходные данные решения участника/студента
	\item код возврата чекера
	\item код возврата решения участника/студента
	\item вердикт чекера
	\item вывод (комментарии) чекера
	\item использованная память
	\item время работы решения участника/студента
\end{itemize}

\textbf{Организаатор контеста} может создать и редактировать контест.

\textbf{Организаатор контеста} может добавлять авторов контест (и удалять их из него).


\end{document}