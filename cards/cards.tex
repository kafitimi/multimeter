\documentclass{book}

% На случай компиляции через PdfLatex
%\usepackage[T1,T2A]{fontenc}
%\usepackage[utf8]{inputenc}

% Поддержка русского языка (переносы и т.п.)
\usepackage[english, russian]{babel}

% Размеры и положение страницы, поля
\usepackage[a6paper, landscape, margin=10mm]{geometry}

% Пакет для работы со шрифтами
\usepackage{fontspec}

% Теховские лигатуры
\defaultfontfeatures[\rmfamily, \sffamily]{Ligatures={TeX}}

% Основным шрифтом используем паратайповский шрифт с засечками
\setmainfont[
	BoldFont={[PT_Serif-Web-Bold.ttf]}
	, BoldItalicFont={[PT_Serif-Web-BoldItalic.ttf]}
	, ItalicFont={[PT_Serif-Web-Italic.ttf]}
	]{[PT_Serif-Web-Regular.ttf]}

% Паратайповский шрифт без засечек
\setsansfont[
	BoldFont={[PT_Sans-Web-Bold.ttf]}
	, BoldItalicFont={[PT_Sans-Web-BoldItalic.ttf]}
	, ItalicFont={[PT_Sans-Web-Italic.ttf]}
	]{[PT_Sans-Web-Regular.ttf]}

% не консоласом единым...
\setmonofont[Scale=1.05]{PTM55FT.ttf}

% какая-то чушь про математические шрифты
\usepackage{unicode-math}
\setmathfont{[latinmodern-math.otf]}
\setmathfont[range=\mathit/{greek,Greek}]{latinmodern-math.otf}

% Красная строка
\usepackage{indentfirst}

% Отключаем номера страниц
\pagestyle{empty}

\newcommand{\newcard}[1]{\newpage \section*{#1}}
\begin{document}

\section*{P-01 Саморегистрация}
\textbf{Участник} должен иметь возможность зарегистрироваться в системе, указав свои данные, если это разрешено администратором.


\newcard{P-02 Сумма баллов}
\textbf{Участник} может видеть свой максимальный балл по каждой задаче контеста и, если разрешено, таблицу баллов для всех участников.


\newcard{P-03 Баллы каждой посылки}
\textbf{Участник} может в зависимости от настроек подзадачи видеть:
\begin{itemize}\setlength{\itemsep}{0pt}
\item общий балл по подзадаче.
\item общий балл по подзадаче либо номер и результат выполнения первого непройденного теста.
\item результат выполнения по каждому тесту.
\end{itemize}


\newcard{P-04 Отправка решения}
\textbf{Участник} может отправить исходный код своего решения в виде файла или ввести его в форму на сайте. При этом он должен указать язык программирования (по умолчанию на странице выбран язык последней отправки).


\newcard{P-05 Страница контеста }
\textbf{Участник} может получать на странице контеста список задач, информацию о максимальном балле по каждой задаче, а также (?) файл с условиями и файл (?) с правилами проведения.


\newcard{P-06 Список контестов }
\textbf{Участник} может выбрать контест из списка доступных ему, при этом ему видно время, оставшееся до окончания контеста, либо дата/время окончания длязавершившихся.


\newcard{P-07 Сброс пароля}
\textbf{Участник} может сбросить свой пароль, получив ссылку сброса на email.


\newcard{P-08 Вопросы в контесте}
\textbf{Участник} может задавать вопросы организаторам контеста, в котором он участвует.


\newcard{P-09 Подглядывать в тесты}
\textbf{Участник} может в зависимости от настроек подзадачи в контесте посмотреть содержание теста, ответа своей посылки, коды выхода своей посылки и чекера, затраченное время и память.

% Автор задачи

\newcard{A-01 Создание и редактирование задач}
\textbf{Автор} может создавать, просматривать, редактировать и удалять задачи. 
\textbf{Автор} должен указать её название.
\textbf{Автор} может указать теги для быстрого поиска.
Система должна автоматически фиксировать дату её модификации.


\newcard{A-02 Условия задачи}
При редактировании задачи \textbf{автор} может ввести:
\begin{itemize}\setlength\itemsep{0pt}
	\item текст условия;
	\item формат входных данных;
	\item формат выходных данных;
	\item примечания;
	\item описание системы оценивания;
	\item разбор;
	\item теги;
\end{itemize}
При этом можно ссылаться на файлы задачи.


\newcard{A-03 Файлы задачи}
\textbf{Автор} задачи может добавлять, просматривать, скачивать и удалять файлы задачи. При этом он может указать тип файла:
\begin{itemize}\setlength{\itemsep}{0pt}
	\item иллюстрация
	\item решение (одно на задачу)
	\item чекер (один на задачу)
	\item генератор тестов
	\item валидатор тестов
\end{itemize}


\newcard{A-04 Подзадачи}
\textbf{Автор} может создавать, просматривать и удалять подзадачи. 


\newcard{A-05 Способ начисления баллов}
\textbf{Автор} должен выбрать для каждой подзадачи способ начисления баллов:
\begin{itemize}\setlength{\itemsep}{0pt}
	\item все или ничего
	\item пропорционально (независимо за каждый тест)
\end{itemize}


\newcard{A-06 Способ отображения результатов}
\textbf{Автор} должен выбрать для каждой подзадачи способ отображения результатов:
\begin{itemize}\setlength{\itemsep}{0pt}
	\item общий балл за подзадачу
	\item общий балл за подзадачу либо номер первого непройденного теста с результатом его проверки
	\item общий балл за подзадачу и результат проверки на каждом тесте
\end{itemize}


\newcard{A-07 Лимиты времени и памяти}
\textbf{Автор} должен указать:
\begin{itemize}\setlength{\itemsep}{0pt}
	\item лимит времени
	\item лимит на используемую память
\end{itemize}


\newcard{A-08 Тесты}
\textbf{Автор} может загружать, скачивать и удалять входные данные тестов.
\textbf{Автор} может создать, просмотреть и отредактировать первые 512 байт входных данных теста.


\newcard{A-09 Генерация входных данных тестов}
\textbf{Автор} может сгенерировать входные данные тестов с помощью генератора тестов. Для этого он должен написать в системе скрипт генерации, в котором могут быть строки вида:
\begin{itemize}\setlength{\itemsep}{0pt}
	\item \texttt{генератор [параметры] > 1}
	\item \texttt{генератор [параметры] > \$}
	\item \texttt{генератор [параметры] > \{индексы\}}
\end{itemize}
где \texttt{генератор} — это имя генератора тестов, \texttt{параметры} — необязательные параметры, \texttt{\$} — автоматически подставляемый номер теста, а \texttt{индексы} — диапазон индексов номеров тестов.


\newcard{A-10 Генерация выходных данных тестов}
\textbf{Система} должна сгенерировать выходные данные тестов (если таковых не существует) при валидации задачи.


\newcard{A-11 Копирование задач}
\textbf{Автор} может создать копию любой задачи к которой он имеет доступ.


\newcard{A-12 Просмотр списка и поиск задач}
\textbf{Автор} может просматривать список задач:
\begin{itemize}\setlength{\itemsep}{0pt}
	\item свои задачи
	\item задачи контестов в которых он участвует 
\end{itemize}

\textbf{Автор} может сортировать и фильтровать задачи:
\begin{itemize}\setlength{\itemsep}{0pt}
	\item по названию
	\item по автору
	\item по тегам
	\item по дате модификации 
\end{itemize}


\newcard{A-13 <<Прекрасное далеко...>>}
\textbf{Автор} может импортировать / экспортировать задачи. Система должна импортировать задачи в формате Polygon.


% Организатор контеста

\newcard{O-01 Редактирование контекста}
\textbf{Организатор контеста} может создать и редактировать контест. Обязательными реквизитами контеста являются
\begin{itemize}\setlength{\itemsep}{0pt}
	\item полное название;
	\item краткое название;
	\item дата и время начала;
	\item дата и время окончания;
	\item выбор правил зачета (личный, командный, лично-командный)
\end{itemize}


\newcard{O-02 Дополнительная информация о тестировании}
\textbf{Организатор контеста} может в любой момент разрешить просмотр отладочной информации по задаче:
\begin{itemize}\setlength{\itemsep}{0pt}
	\item входные данные теста (можно скачать тест)
	\item выходные данные теста (можно скачать ответ)
	\item выходные данные решения участника/студента
	\item код возврата чекера
	\item код возврата решения участника/студента
	\item вердикт чекера
	\item вывод (комментарии) чекера
	\item использованная память
	\item время работы решения участника/студента
\end{itemize}


\newcard{O-03 Таблица результатов}
	Общая таблица результатов всех участников контесте 
\begin{itemize}\setlength{\itemsep}{0pt}
	\item всегда видна организатору контеста
	\item видна гостю, если организатор установил флажок <<гостевой доступ>>
	\item видна участникам, если организатор установил флажок <<публиковать таблицу результатов для участников>>
\end{itemize}
	Если организатор указал время заморозки таблицы, то она перестает обновляться для участников и гостей в момент заморозки.


\newcard{O-04 Гостевой доступ}
	Если организатор установил флажок <<гостевой доступ>>, как правило, для болельщиков/учителей участников, то система позволяет войти без пароля и посмотреть таблицу результатов.
	
	Если организатор установил флажок <<публиковать тесты для участников и гостей>>, то гости могут видеть и скачивать тесты.


\newcard{О-05 Доступ для участников}
	\textbf{Участники} не могут видеть условия раньше времени начала контеста. 
	
	\textbf{Участники} не могут отправлять решения после окончания контеста. 
	
	В любой момент \textbf{организатор} может временно заблокировать доступ участников к контесту. 


\newcard{О-06 Публикация материалов соревнования}
	\textbf{Организатор} контеста может опубликовать тесты и чекеры всех задач для участников и гостей.


\newcard{О-07 Добавление и удаление авторов}
	\textbf{Организатор контеста} может добавлять авторов в контест (и удалять их из него). 


\newcard{О-08 Просмотр и сортировака задач в контесте}
	\textbf{Организатор контеста} может просматривать добавленные авторами в контест задачи и менять порядок их следования.


\newcard{О-09 Объявления}
	\textbf{Организатор контеста} должен иметь возможность создать текстовое оповещение, которое смогут увидеть все участники контеста.


\newcard{О-10 Личный зачет}
	В случае, если организатор выбрал для проведения контеста правила личного зачета, тогда в таблицу результатов должен попасть максимальный балл полученный решением участника по каждой задаче.

	Участники сортируются по убыванию суммы баллов за все задачи. При равенстве баллов участники сортируются по алфавиту.


\newcard{О-11 Командный зачет}
	В случае, если организатор выбрал для проведения контеста правила командного зачета, тогда в таблице результатов по каждой задаче должно отображаться:
	\begin{itemize}
	\item факт решения задачи (знаком «+» или «­–»);
	\item количество неудачных попыток решения задачи;
	\item если задач решена, то показывается время отправки полного решения.
	\end{itemize}
	
	В качестве итоговых колонок расчитаваются:
	\begin{itemize}
		\item количество решенных задач;
		\item суммарное время, потраченное на успешное решение задач, плюс 20 минут штрафа за каждую неудачную попытку решения (неудачные попытки, не приведшие к решению задачи, не учитываются).
	\end{itemize} 
	
	По умолчанию участники сортируются по убыванию количества решенных задач, а при равенстве количества решенных задач участники сортируются по возрастанию суммарного времени. Организатор контеста может указать алфавитный порядок выдачи результатов (для лабораторных работ).


\newcard{О-12 Лично-командный зачет}
	В случае, если организатор выбрал для проведения контеста правила лично-командного зачета, тогда в таблицу результатов должен попасть максимальный балл, полученный решением участника по каждой задаче.

	Помимо этого, каждый участник должен указать свою команду. Итоговые баллы участников суммируются покомандно в отдельной таблице результатов для команд.

	Обе таблицы сортируются по убыванию суммы баллов за все задачи. При равенстве баллов таблицы сортируются по алфавиту.
	
\end{document}
