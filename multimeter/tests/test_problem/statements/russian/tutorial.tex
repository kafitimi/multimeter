\begin{tutorial}{Пожарная охрана}

Для решения задачи будем использовать метод динамического программирования. Пусть $A_i$~--- это минимальные затраты на обеспечение пожарной безопасности населенных пунктов с~1 по $i$, при условии, что в~пункте $i$ находится пожарная станция. Очевидно, что $A_1=C_1$. Если станция находится в~пункте $i$, то нужно, чтобы предыдущая станция находилась на расстоянии не превышающем $2k+1$. Поэтому $A_i = \min(A_{\max(i-2k-1,1)}, A_{\max(i-2k-1,1)+1}, \dots, A_{i-1}) + C_i$. Ответом задачи будет $\min(A_{\max(n-k,1)}, A_{\max(n-k,1)+1}, \dots, A_n)$.

\end{tutorial}
