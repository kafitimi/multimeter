\begin{problem}{Пожарная охрана}{fire.in}{fire.out}{2 секунды}{64 мегабайта}

В одной республике все населенные пункты расположены вдоль одной единственной 
дороги. Поскольку большинство построек в них деревянные, то одной из важных задач 
является организация пожарной охраны. Для этого необходимо построить вдоль дороги
несколько пожарных станций, в которых будут базироваться пожарные машины.

Станции можно построить только в населенных пунктах, причем затраты на постройку
пожарной станции в разных населенных пунктах могут быть различны. Будем считать, что 
населенные пункты пронумерованы от одного конца дороги к другому последовательно 
числами от 1 до $n$, а стоимость постройки пожарной станции в населенном пункте
с номером $i$ равна $c_i$.

Населенные пункты расположены вдоль дороги достаточно равномерно, поэтому будем
считать, что расстояние между любыми двумя соседними населенными пунктами пожарная машина
проезжает за один час. Если от момента возникновения пожара до момента приезда
пожарной машины проходит более $k$ часов, то горящий объект спасти не удается.
Поэтому от любого населенного пункта (вне населенных пунктов пожары не возникают) до 
ближайшей пожарной станции путь должен занимать не более $k$ часов.

Определите наименьшую сумму затрат на строительство пожарных станций.


\InputFile
Первая строка входного файла содержит два числа $n$ и $k$ ($1 \le n \le 10000$, $0 \le k \le 100$)~--- количество 
населенных пунктов, находящихся на дороге и максимальное допустимое время прибытия 
пожарной машины $k$.
Вторая строка содержит $n$ чисел $c_i$ ($0 \le c_i \le 10^{6}$)~--- стоимость постройки 
пожарной станции в населенном пункте с номером $i$.


\OutputFile
В единственной строке выходного файла выведите одно число --- минимальную суммарную стоимость 
постройки всех пожарных станций.


\Scoring
\SubtaskOne
Дополнительное ограничение:
$n \le 10$

Баллы начисляются только в том случае, если все тесты данной подзадачи и тесты из примера пройдены.

\SubtaskTwo
Дополнительное ограничение:
$k \le 1$

Баллы начисляются только в том случае, если все тесты данной подзадачи и тесты из примера пройдены.

\SubtaskThree
Дополнительных ограничений нет.

Баллы начисляются только в том случае, если все тесты данной подзадачи и тесты из примера пройдены.


\Examples

\begin{example}
\exmp{4 2
2 4 3 2
}{3}%
\exmp{3 1
2 3 2
}{3}%
\end{example}

\end{problem}

